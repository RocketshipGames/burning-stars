%%----------------------------------------------------------------------
%% Decorations
%%----------------------------------------------------------------------

\newcommand{\teaser}[1]{\centerline{\emph{#1}}}

%%----------------------------------------------------------------------
%% Deployment Zones
%%----------------------------------------------------------------------

\newenvironment{tablesetup}
{\missionheading{Table Setup}}
{}

\newcommand{\dawnofwar}%
{Deployment zones are \textbf{Dawn of War}, as on page~131 of the
\emph{Warhammer 40,000} rulebook (12'' long edges).}

\newcommand{\hammerandanvil}%
{Deployment zones are \textbf{Hammer and Anvil}, as defined on
  page~131 of the main rulebook (24'' short edges).}

\newcommand{\vanguardstrike}%
{Deployment zones are \textbf{Vanguard Strike}, as defined on page~131
  of the main \emph{Warhammer 40,000} rulebook.  Vanguard Strike may
  be approximated by deploying within a 33.5'' x 50'' table corner
  triangle.  The player that wins the zone roll off may pick any of
  the four corners, and the other player takes that diagonally
  opposite.}

\newcommand{\quartered}%
{Deployment zones are the rectangles in each table corner~12'' in from
  the long edge and~24'' in from the short edges.  The player that
  wins the zone roll off picks either pair of \emph{diagonally
    opposite} corners as their deployment zone and a long table edge
  as their player edge.  The other player takes the other pair of
  diagonally opposite corners and opposite long edge.}

%%----------------------------------------------------------------------
%% Mission Rules
%%----------------------------------------------------------------------

\newenvironment{missionrules}
{
\missionheading{Mission Specific Rules}

The following mission specific rules apply, in addition to those
applied to all missions in this packet.
}
{
}

\newcommand{\nightfighting}%
{\missionsubheading{Night Fighting.}  If either player opts for Night
  Fighting before any deployment begins, on a single D6 of 4+ all
  units have Stealth throughout Turn 1.}

\newcommand{\nightfalls}%
{\missionsubheading{Night Falls.}  If either player opts for Night
  Falls before any deployment begins, on a single~D6 of~4+ all units
  except superheavy vehicles and gargantuan creatures have Stealth
  throughout Turns~5,~6, and~7.}


%%----------------------------------------------------------------------
%% Scoring
%%----------------------------------------------------------------------

\newenvironment{scoring}{
\missionheading{Scoring}

% This mission is scored by objectives achieved, as follows.
}
{

\missionsubheading{Secondary Objectives.}

After deployment, both players simultaneously choose and reveal a
personal secondary objective from the options available for their
campaign goal.  Any necessary selections are also chosen and revealed
then unless noted otherwise.  \underline{No more than~6 victory points
  may be earned via these.}

\missionsubheading{Tertiary Objectives.}  As given in the overall
Common Rules section of this packet.
}

\newenvironment{primaries}
{
\missionsubheading{Primary Objectives.}  
}
{}


%%----------------------------------------------------------------------
%% Secondaries
%%----------------------------------------------------------------------

\newenvironment{secondaries}
{
}
{
}

\newcommand{\seekanddestroy}%
{\item \textit{Seek and Destroy.}  Choose and declare a Battlefield
  Role other than Troop.  Score~2 victory points for each enemy unit
  of this role completely destroyed or falling back at the end of the
  game.}

\newcommand{\securepositions}%
{\item \textit{Secure Positions.}  For each table quadrant, secretly
  select a terrain piece at least~50\% inside it, recording your
  selections unambiguously.  Reveal these at game end and score~2
  victory points for each that you control, treating them as objective
  markers.  Note that this means a single unit cannot claim both a
  primary objective marker and a terrain piece, nor multiple pieces
  simultaneously.}

\newcommand{\breachpoints}%
{\item \textit{Breach Points.}  Choose two terrain pieces at least
  partially in the opposing deployment zone.  Do not declare these
  now, but do secretly record your selection unambiguously!  Reveal
  these at game end and score~3 victory points for each piece that you
  control, treating them as objective markers.  Note that this means a
  single unit cannot claim both a primary objective marker and a
  terrain piece simultaneously.}

\newcommand{\seizeground}%
{\item \textit{Seize Ground.}  Choose two terrain pieces not in your
  deployment zone.  Do not declare these now, but do secretly record
  your selection unambiguously!  Reveal these at game end and score~3
  victory points for each piece that you control, treating them as
  objective markers.  Note that this means a single unit cannot claim
  both a primary objective marker and a terrain piece simultaneously.}

\newcommand{\reconnaissance}%
{\item \textit{Reconnaissance.}  At game end, score~2 victory points
  for each friendly scoring unit with the Scout or Infiltrate USR
  completely within 12'' of your opponent's table edge.}

\newcommand{\meatgrinder}%
{\item \textit{Meatgrinder.}  Score~1 victory point for each opposing
  Troop unit completely destroyed or falling back at the end of the
  game.}

\newcommand{\hullbreaker}%
{\item \textit{Hullbreaker.}  Score~1 victory point for each opposing
  vehicle completely destroyed or monstrous creature removed as a
  casualty.}

\newcommand{\assassination}%
{\item \textit{Assassination.}  Score~1 victory point for each
  opposing character model removed as a casualty or falling back at
  the end of the game.  Note that this is not limited to just
  independent characters.}

\newcommand{\interrogation}%
{\item \textit{Interrogation.}  Score~1 victory point for each
  opposing character model removed as a casualty in close combat.  In
  addition, whenever an opposing character model is removed as a
  casualty by any means, put a secondary objective marker in its
  place.  You score~1 victory point for each such marker controlled at
  game end.  Note that neither of these criteria are limited to just
  independent characters.}

\newcommand{\frontline}%
{\item \textit{Frontline.} Place three secondary objective markers
  within~18'' of your opponent's table edge, 6'' from all table edges
  and 12'' from all objective markers or as far apart as possible. At
  game end, score 3 victory points for each of these markers you
  control.}

\newcommand{\controlthefield}%
{\item \textit{Control the Field.}  Each table quarter in which you
  have a scoring unit and your opponent does not, or you have an
  Objective Secured Unit and your opponent does not, is worth 2
  victory points at game end.}

\newcommand{\breaktheirback}%
{\item \textit{Break Their Back.}  At game end, each enemy unit that
  has been eliminated, is falling back, or has at most~25\% of its
  starting models remaining is broken.  Earn~2 victory points per
  quartile if at least 25\%, 50\%, and 75\% of your opponent's army by
  units is broken.}

\newcommand{\stalwart}%
{\item \textit{Stalwart.} Score~2 victory points for each objective
  marker held at game end.}

\newcommand{\holdthefield}%
{\item \textit{Hold The Field.} Score~2 victory points for every~2
  objective markers held at game end.}

\newcommand{\overrun}%
{\item \textit{Overrun.}  At game end, count how many scoring units
  you have at least partially within your opponent's half of the
  table, and how many scoring units your opponent has at least
  partially within your half of the table.  If your number is higher,
  score~2 victory points for each point of difference.}

\newcommand{\majoritycontrol}%
{\item \textit{Majority Control.}  At the end of each game turn,
  score~1 victory point if you control more objective markers than
  your opponent.  Score an additional point if you control more than
  half of the markers.}


%%----------------------------------------------------------------------
%% Maelstrom
%%----------------------------------------------------------------------

\newcommand{\maelstrom}{%
  \missionsubheading{The Storm.}%
  At the start of each player turn, the active player draws tactical
  objectives until they have as many in play as the current turn
  number.

  \maelstromrules
}

\newcommand{\standingorders}{%
  \missionsubheading{Standing Orders.}%
  At the start of your turns, draw tactical
  objectives until you have a total of six in play.

  \maelstromrules
}

\newcommand{\maelstromrules}{%
  \missionsubheading{Maelstrom.}%
  To draw a tactical objective, roll a~D66 and consult your tactical
  objective table, attached at the end of this mission packet.  If
  that objective is already in play for you, has been achieved, or is
  scratched off, roll again.  Similarly, if that objective would be
  provably impossible to score, e.g., your opponent has no characters
  remaining, roll again.  Once a valid objective has been rolled, mark
  it as in play.

  Targets cannot be nominated or chosen for a tactical objective
  marked with a $\dagger$ that have already been chosen for a
  $\dagger$ objective you have in play.

  \smallskip%
  At the end of your turns, check the requirements for each tactical
  objective you have in play.  For each fixed-value objective met,
  mark it as achieved and score the associated value in \emph{mission
    points} (n.b.: not \emph{victory points}).  Tactical objectives
  with a value of X may be kept in play as long as you wish.  At the
  end of any of your turns while in play they may be marked as
  achieved and scored as indicated.  Once achieved, objectives are no
  longer considered in play and cannot be put in play or scored again.

  Multiple objectives can be scored in a turn, caveat that you cannot
  achieve multiple tactical objectives with the same exact title in
  the same turn using the same marker(s) or unit(s).  E.g., to score
  both Storm objectives at once, you would need to simultaneously
  control two separate markers in the enemy deployment zone.

  At the end of your turn you may scratch out one of your tactical
  objectives in play to remove it from play.

  \smallskip%
  Tactical objectives in play, achieved, and scratched out are not
  secret.}

\newcommand{\maelstromscoring}
{At game end, compare mission points earned through tactical
objectives achieved and award victory points to the higher and lower
scorer as follows:

%\definecolor{Gray}{gray}{0.9}
%\definecolor{DGray}{gray}{0.75}
\newcolumntype{a}{>{\columncolor{gray!25}}c}
\newcolumntype{b}{>{\columncolor{gray!50}}r}
\bigskip\centerline{\setlength{\tabcolsep}{12pt}%
\begin{tabular}{|b|c|a|c|a|c|a|}
\hline
{\bf Difference}     & 0     & 1--2  & 3--4  & 5--6  & 7     & 8+\\
{\bf Victory Points} & 4 / 4 & 5 / 4 & 6 / 3 & 7 / 2 & 8 / 1 & 9 / 0\\
\hline
\end{tabular}}}
